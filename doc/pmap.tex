\append{Тестирование функции pmap}

Функция $pmap$ действует абсолютно так же, как и $map$, за исключением того, что функция $f$ применяется к элементам параллельно. Её очень удобно использовать для организации барьерной \textbf{fork-join} синхронизации. Так же, как и $map$, $pmap$ возвращает ленивую последовательность и порядок применения функции не стандартизирован.

Не стоит забывать, что $pmap$ полезна только если $f$ -- относительно долгая операция, на порядки превышающая расходы на создание потоков. Иначе, функция $pmap$ работает гарантированно дольше однопоточного аналога.

\inputminted[fontsize=\small, firstline=42,lastline=57]{clj}{src/book.clj}

\begin{center}
  \begin{tabular}{|c|c|}
    \hline
    $f$ & Время, \textit{msecs} \\
    \hline
    $map$ & 5465.595298  \\
    \hline
    $pmap$ & 3760.315602  \\
    \hline
  \end{tabular}
\end{center}


